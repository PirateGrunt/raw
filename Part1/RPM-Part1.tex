\documentclass[xcolor=dvipsnames]{beamer}
\usepackage[]{graphicx}\usepackage[]{color}
%% maxwidth is the original width if it is less than linewidth
%% otherwise use linewidth (to make sure the graphics do not exceed the margin)
\makeatletter
\def\maxwidth{ %
  \ifdim\Gin@nat@width>\linewidth
    \linewidth
  \else
    \Gin@nat@width
  \fi
}
\makeatother

\definecolor{fgcolor}{rgb}{0.345, 0.345, 0.345}
\newcommand{\hlnum}[1]{\textcolor[rgb]{0.686,0.059,0.569}{#1}}%
\newcommand{\hlstr}[1]{\textcolor[rgb]{0.192,0.494,0.8}{#1}}%
\newcommand{\hlcom}[1]{\textcolor[rgb]{0.678,0.584,0.686}{\textit{#1}}}%
\newcommand{\hlopt}[1]{\textcolor[rgb]{0,0,0}{#1}}%
\newcommand{\hlstd}[1]{\textcolor[rgb]{0.345,0.345,0.345}{#1}}%
\newcommand{\hlkwa}[1]{\textcolor[rgb]{0.161,0.373,0.58}{\textbf{#1}}}%
\newcommand{\hlkwb}[1]{\textcolor[rgb]{0.69,0.353,0.396}{#1}}%
\newcommand{\hlkwc}[1]{\textcolor[rgb]{0.333,0.667,0.333}{#1}}%
\newcommand{\hlkwd}[1]{\textcolor[rgb]{0.737,0.353,0.396}{\textbf{#1}}}%

\usepackage{framed}
\makeatletter
\newenvironment{kframe}{%
 \def\at@end@of@kframe{}%
 \ifinner\ifhmode%
  \def\at@end@of@kframe{\end{minipage}}%
  \begin{minipage}{\columnwidth}%
 \fi\fi%
 \def\FrameCommand##1{\hskip\@totalleftmargin \hskip-\fboxsep
 \colorbox{shadecolor}{##1}\hskip-\fboxsep
     % There is no \\@totalrightmargin, so:
     \hskip-\linewidth \hskip-\@totalleftmargin \hskip\columnwidth}%
 \MakeFramed {\advance\hsize-\width
   \@totalleftmargin\z@ \linewidth\hsize
   \@setminipage}}%
 {\par\unskip\endMakeFramed%
 \at@end@of@kframe}
\makeatother

\definecolor{shadecolor}{rgb}{.97, .97, .97}
\definecolor{messagecolor}{rgb}{0, 0, 0}
\definecolor{warningcolor}{rgb}{1, 0, 1}
\definecolor{errorcolor}{rgb}{1, 0, 0}
\newenvironment{knitrout}{}{} % an empty environment to be redefined in TeX

\usepackage{alltt}
\newcommand{\SweaveOpts}[1]{}  % do not interfere with LaTeX
\newcommand{\SweaveInput}[1]{} % because they are not real TeX commands
\newcommand{\Sexpr}[1]{}       % will only be parsed by R


%\documentclass{beamer}
\usepackage[dvipsnames]{xcolor}

\usefonttheme[onlymath]{serif}
\usetheme{Berlin}
\setbeamercolor{title}{fg=black}
\setbeamercolor{frametitle}{fg=black}

\setbeamertemplate{itemize items}[circle] 
\setbeamercolor{itemize item}{fg=black} 


\begin{document}



\begin{frame}{Installation}
  Let's get this done quick. If you don't have R and R Studio installed, today will be a long day.
\end{frame}

\begin{frame}{What the heck is R?}
  While some folks are getting software sorted out, we'll have a brief review of the history of R. 
\end{frame}

\begin{frame}[fragile]
\begin{knitrout}
\definecolor{shadecolor}{rgb}{0.969, 0.969, 0.969}\color{fgcolor}\begin{kframe}
\begin{alltt}
\hlkwd{set.seed}\hlstd{(}\hlnum{1234}\hlstd{)}
\hlstd{N} \hlkwb{=} \hlnum{100}
\hlstd{e} \hlkwb{=} \hlkwd{rnorm}\hlstd{(N,} \hlkwc{mean} \hlstd{=} \hlnum{0}\hlstd{,} \hlkwc{sd} \hlstd{=} \hlnum{1}\hlstd{)}
\hlstd{B0} \hlkwb{=} \hlnum{5}
\hlstd{B1} \hlkwb{=} \hlnum{1.5}

\hlstd{X1} \hlkwb{=} \hlkwd{rep}\hlstd{(}\hlkwd{seq}\hlstd{(}\hlnum{1}\hlstd{,} \hlnum{10}\hlstd{),} \hlnum{10}\hlstd{)}
\hlstd{Y} \hlkwb{=} \hlstd{B0} \hlopt{+} \hlstd{B1} \hlopt{*} \hlstd{X1} \hlopt{+} \hlstd{e}

\hlstd{myFit} \hlkwb{=} \hlkwd{lm}\hlstd{(Y} \hlopt{~} \hlstd{X1)}
\end{alltt}
\end{kframe}
\end{knitrout}

\end{frame}
\end{document}
